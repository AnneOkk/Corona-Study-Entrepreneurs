%%
%% This is file `./samples/longsample.tex',
%% generated with the docstrip utility.
%%
%% The original source files were:
%%
%% apa7.dtx  (with options: `longsample')
%% ----------------------------------------------------------------------
%% 
%% apa7 - A LaTeX class for formatting documents in compliance with the
%% American Psychological Association's Publication Manual, 7th edition
%% 
%% Copyright (C) 2019 by Daniel A. Weiss <daniel.weiss.led at gmail.com>
%% 
%% This work may be distributed and/or modified under the
%% conditions of the LaTeX Project Public License (LPPL), either
%% version 1.3c of this license or (at your option) any later
%% version.  The latest version of this license is in the file:
%% 
%% http://www.latex-project.org/lppl.txt
%% 
%% Users may freely modify these files without permission, as long as the
%% copyright line and this statement are maintained intact.
%% 
%% This work is not endorsed by, affiliated with, or probably even known
%% by, the American Psychological Association.
%% 
%% ----------------------------------------------------------------------
%% 
\documentclass[man]{apa7}

\usepackage{lipsum}
\usepackage[american]{babel}
\usepackage{setspace}
\usepackage[compact]{titlesec}

\parskip 0.0in 
% \titlespacing*{\section}
% {0pt}{0.0ex plus 1ex minus .0ex}{0.0ex plus .0ex}
% \titlespacing*{\subsection}
% {0pt}{0.0ex plus 1ex minus .0ex}{0.0ex plus .0ex}

\titlespacing{\section}{0pt}{2ex}{1ex}
    \titlespacing{\subsection}{0pt}{1ex}{0ex}
    \titlespacing{\subsubsection}{0pt}{0.5ex}{0ex}

\usepackage{csquotes}
\usepackage[style=apa,sortcites=true,sorting=nyt,backend=biber]{biblatex}
\DeclareLanguageMapping{american}{american-apa}
\addbibresource{CoronaEntre.bib}

\title{Financial Strain as a Predictor of Entrepreneurs' Life Satisfaction and the Intention to Quit Entrepreneurship during the COVID-19 Pandemic: The Role of Coping}
\shorttitle{Entrepreneurs' intention to quit during corona}

\author{Anne-Kathrin Kleine, Antje Schmitt \& Barbara Wisse}
\affiliation{University of Groningen}

\leftheader{Weiss}

\abstract{}

\keywords{corona crisis, entrepreneurship, financial strain, coping}

\authornote{
   \addORCIDlink{Anne-Kathrin Kleine}{0000-0003-1919-2834}

  Correspondence concerning this article should be addressed to Anne-Kathrin Kleine, Faculty of Social and Behavioral Sciences, Department of Organizational Psychology, University of Groningen, Grote Kruisstraat 2/1, 9712 TS Groningen.  E-mail: a.k.kleine@rug.nl}

\begin{document}
\maketitle

Measures to curb the spread of the SARS-CoV-2 virus that causes COVID-19 disease - such as lockdowns and social distancing - have lead to the rapid contraction of global economies \parencite[e.g.,][]{Brown2020, Thorgren2020}. 
The negative psychological impact of the current crisis may be expected to affect both wage workers and the self-employed.\footnote{Note that we refer to entrepreneurs as individuals working self-employed \parencite[see, e.g.,][]{Gorgievski2016a}. We use these terms interchangeably.}
In contrast to employed workers, it is difficult for the self-employed to verify work behavior, which means they have long been left out from most coronavirus job retention schemes \parencite{Blundell2020, Bertschek2020}.
Available financial aids may help buffering the immediate negative consequences and help self-employed to pay ongoing costs. 
However, emergency loans may pile up, thus constituting a heavy financial burden over time. 
Even if debts can be paid off, they hinder necessary investment. 
Out of fear of high debt, some self-employed may hesitate to accept emergency loans.
In combination with a lack of investment these scenarios may increase the risk of entrepreneurs experiencing financial hardship. 
In a survey of 16,000 German solo self-employed, every fourth saw a high probability of having to give up self-employment in the next twelve months. 
The main reason was a massive drop in sales during the current crisis. 
Almost 60 percent indicated that their monthly turnover at the time of the survey in April 2020 had dipped by more than 75 percent \parencite{Bertschek2020}. \par 

The experience of financial hardship may have severe negative consequences for entrepreneurs' well-being \parencite[e.g.,][]{Annink2016a}. 
Particularly during turbulent times, such as during the current COVID-19 crisis, relative to wage workers, the self-employed experience greater psychological distress through self-reported financial insecurity \parencite{Patel2020a}.
Moreover, the experience of financial strain has been associated with greater quit intentions that may elicit further loss spirals \parencite{Gorgievski2010a}.
Specifically, building on conservation of resources theory, \textcite{Gorgievski2010a} found that among self-employed farmers, financial strain was associated with the intention to quit one's business, which, in turn, predicted the deterioration of financial problems one year later. \par 

In the current study, we build on and extend previous research findings by investigating problem-focused coping as a means to buffer the negative consequences of financial strain for entrepreneurs' well-being and quit intentions. 
Coping styles are generalized patterns of attempts to overcome adverse conditions \parencite[e.g.,][]{Carver1994}. 
People who possess a problem-focused coping style tend to focus on pragmatic ways of dealing with problems.
That is, they face and actively manage the problem at hand, by making a plan of action \parencite{Folkman1984, Folkman2004}.
Qualitative research findings indicate that problem-focused coping strategies help entrepreneurs to settle anxieties in post-disaster consequences \parencite{Fang2020}. 
The main goal of the current study is to empirically investigate the beneficial role of problem-focused coping as a coping style that buffers the adverse effects of financial strain on entrepreneurs' intention to quit their business and overall life satisfaction. \par 

Business crises result from a weakening or degeneration that disrupts the business' normal functioning \parencite{Williams2017a}.
This weakening or degeneration may result from low-probability external events, like pandemics or natural disasters, or daily disturbances, such as customer dissatisfaction \parencite{Doern2019}.
Studying the effects of financial hardship and problem-focused coping on entrepreneurs' intention to quit and life satisfaction in the context of the current COVID-19 pandemic allows us to draw general conclusions about how entrepreneurs are affected by and may effectively cope with financial problems during crises. 

\section{Theoretical Background}

According to the OECD, income and wealth are essential components of individual health and well-being \parencite{OECD2013}.
Regular income allows people to satisfy their needs and wealth guarantees them the security to sustain a chosen lifestyle over a longer period of time \parencite{OECD2013}.  
Subjective financial hardship occurs when financial strain is perceived or the occurrence of future financial problems is expected \parencite[see][]{Schieman2011}.
The experience of financial hardship relates positively to anxiety, depression, and poor physical health, as well as feelings of shame and guilt \parencite[e.g.,][]{OECD2013, Kahn2006, McDaid2013, Peirce1994, Starrin2009}.  
It has been argued that financial hardship particularly hits those working self-employed \parencite[e.g.,][]{Patel2020a}. 
Indeed, using longitudinal data from the Understanding America Study that was collected during the current COVID-19 pandemic, \textcite{Patel2020a} showed that, relative to wage-workers, the self-employed experience greater psychological distress as a consequence of the fear of running out of money.
Financial problems do not only affect the success of the business, but also make it difficult to fulfill personal and family needs \parencite[e.g., paying off mortgages,][]{Gorgievski2010a}. 
That is, self-employed who experience financial hardship are restricted in their roles both as entrepreneurs and as citizens. 
According to spillover theories, if a person is exposed to unpleasant circumstances in an essential area of their life, the resulting emotional experiences may directly affect that person's life satisfaction as a global evaluation of the quality of their life as a whole \parencite{McDowell2010, Pavot2009, Kantak1992, Pasupuleti2009}.
When basic financial requirements can no longer be met, the idea of shutting down the business to avoid further loss may seem obvious \parencite{Gorgievski2010a}.
Accordingly, the stress associated with financial hardship may increase the intention to quit one's business \parencite[e.g.,][]{Annink2016, Gorgievski2010a}. \par 

In the current study, we extend recent research findings by exploring how a problem-focused coping style may buffer the negative effects of financial hardship on life satisfaction and the intention to quit. 
Coping styles are generalized patterns of attempts to overcome adverse conditions \parencite[e.g.,][]{Carver1994}. 
In this sense, coping styles reflect trait-like dispositions that influence individual responses to stressors \parencite{Carver1994}.
A problem-focused coping style is associated with self-efficacy, i.e., the belief to be able to change an adverse situation by investing personal effort \parencite{Bandura.1986}. 
People who possess a problem-focused coping style tend to focus on pragmatic ways of dealing with problems.
They face and actively manage the problem at hand, by making a plan of action \parencite{Folkman1984, Folkman2004}.
According to Lazarus and colleagues' transactional stress theory \parencite[TST, e.g.,][]{Lazarus.2000, Folkman.1986, Lazarus.1978}, problem-focused coping may not be beneficial for health and well-being under all circumstances. 
Rather, coping effectiveness is dependent on whether it matches the perceived controllability of the stressor. 
Several studies have found support for the assumption that problem-focused coping is beneficial in cases where the adverse event is perceived as being controllable \parencite[e.g.,][]{Felton1984}. 
While the occurrence of disastrous events, such as the COVID-19 pandemic, is beyond our control, entrepreneurs may very well exert some control over financial problems and their consequences.
Accordingly, we argue that a problem-focused coping style may mitigate the negative effects of financial strain on psychological outcomes. 
Indeed, results of interviews with self-employed tourism business owners who suffered from adversity in the context of a natural disaster (earthquakes) indicate that those who focused on controllable consequences of the catastrophe and employed problem-focused coping strategies were able to settle off anxieties \parencite{Fang2020}. \par 

Problem-focused coping in the context of crisis may not only protect well-being and health, but may also influence whether individuals give up (e.g., quit their work) or persist when being faced with controllable work-related stressors \parencite[e.g.,][]{Zellars2004, Cunningham2006}. 
For example, \textcite{Cunningham2006} studied the role of problem-focused coping as a predictor of the intention to quit during phases of organizational change. 
The authors found that the more employees adopted problem-focused coping behaviors, the less likely they were to quit their job.
Problem-focused coping is associated with the tendency to appraise pressing problems as challenging rather than threatening \parencite[e.g.,][]{Folkman.1986}.
That is, rather than focusing on the negative consequences of an adverse situation, a problem-focused coping style allows entrepreneurs to perceive opportunities for growth and change, thus preventing them from giving up and quitting their business.
Indeed, challenge appraisal of a stressful encounter was found to decrease entrepreneurs' exit intentions in the face of adversity \parencite{Zhu2017}.
Drawing from TST and past research findings, we propose the following two hypotheses: \par 

\textit{Hypothesis 1}: Problem-focused coping style moderates the relationship between T1 financial strain and T2 life satisfaction; specifically, high problem-focused coping style buffers the negative effect of T1 financial strain on T2 life satisfaction. \par 

\textit{Hypothesis 2}: Problem-focused coping style moderates the relationship between T1 financial strain and T2 intention to quit one's business; specifically, high problem-focused coping style buffers the negative effect of T1 financial strain on T2 quit intention. \par 

\printbibliography

\end{document}




%% 
%% Copyright (C) 2019 by Daniel A. Weiss <daniel.weiss.led at gmail.com>
%% 
%% This work may be distributed and/or modified under the
%% conditions of the LaTeX Project Public License (LPPL), either
%% version 1.3c of this license or (at your option) any later
%% version.  The latest version of this license is in the file:
%% 
%% http://www.latex-project.org/lppl.txt
%% 
%% Users may freely modify these files without permission, as long as the
%% copyright line and this statement are maintained intact.
%% 
%% This work is not endorsed by, affiliated with, or probably even known
%% by, the American Psychological Association.
%% 
%% This work is "maintained" (as per LPPL maintenance status) by
%% Daniel A. Weiss.
%% 
%% This work consists of the file  apa7.dtx
%% and the derived files           apa7.ins,
%%                                 apa7.cls,
%%                                 apa7.pdf,
%%                                 README,
%%                                 APA7american.txt,
%%                                 APA7british.txt,
%%                                 APA7dutch.txt,
%%                                 APA7english.txt,
%%                                 APA7german.txt,
%%                                 APA7ngerman.txt,
%%                                 APA7greek.txt,
%%                                 APA7czech.txt,
%%                                 APA7turkish.txt,
%%                                 APA7endfloat.cfg,
%%                                 Figure1.pdf,
%%                                 shortsample.tex,
%%                                 longsample.tex, and
%%                                 bibliography.bib.
%% 
%%
%% End of file `./samples/longsample.tex'.
